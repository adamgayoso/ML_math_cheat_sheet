\documentclass[11pt]{article}

\usepackage[textwidth=6.5in, textheight=9.5in]{geometry}

\usepackage{amsfonts}
\usepackage{amssymb,amsmath, amsthm}
\usepackage{siunitx}
\usepackage{url, graphicx}
\usepackage{xspace}

\usepackage{bm} % Define common bold symbols with \bmdefine{cmd}{def}. Use \bm{var} ad hoc in text
\bmdefine{\bmx}{x}
\bmdefine{\bmA}{A}
\bmdefine{\bma}{a}
\bmdefine{\bmy}{y}
\bmdefine{\bmB}{B}
\bmdefine{\bmmu}{\mu}
\bmdefine{\bmSigma}{\Sigma}

%%% Destinguished symbols
\def\R{{\mathbb{R}}}
\def\Given{{\,|\,}}
\DeclareMathOperator{\E}{\mathbb{E}}
\DeclareMathOperator{\Var}{Var}
\DeclareMathOperator{\Cov}{cov}
\def\Def{{\stackrel{\text{def}}{=}}}
\def\iid{{\stackrel{\text{iid}}{\sim}}}
\def\Ind{{\stackrel{\text{ind}}{\sim}}}
\DeclareMathOperator{\Trace}{tr}

%%% Math operator shortcuts
\newcommand*{\Inner}[2]{\left\langle #1, #2 \right\rangle}
\newcommand*{\Prox}[2]{\operatorname{prox}_{#1}(#2)}
\newcommand*{\fracbig}[2]{\left( \frac{#1}{#2} \right)}

%%% Other special abreviation defs

\renewcommand{\thesubsection}{\thesection.\alph{subsection}}
\renewcommand{\thesubsubsection}{\thesubsection.\roman{subsubsection}}

\usepackage[colorlinks,linkcolor=blue]{hyperref}
\usepackage{cleveref} % Needs to be last

%opening
\title{ML \& LinAlg Math Cheat Sheet}

\begin{document}

\maketitle
\tableofcontents

\section{Notation}
Vectors are column vectors denoted by lower-case bolded variables, such that 
\[\bmx = \left[ \begin{split}
x_1 \\ \vdots \\ x_N 
\end{split} \right]. \]
A row vector is denoted $\bmx^\top = [x_1 \ldots x_N]$.
A matrix is indicated by a bolded upper-case variable, such that an $N \times M$ matrix is
\begin{align*}
\bmA = \{a_{ij}\} = [\bma_1 \cdots \bma_M ] =
\left[\begin{array}{c}
\bma_1^\top \\
\vdots \\
\bma_N^\top
\end{array}\right]
=
\left[\begin{array}{ccc}
a_{1,1}^\top & \cdots & a_{1, M} \\
\vdots & \ddots & \vdots \\
a_N^\top & \cdots & a_{N, M}
\end{array}\right].
\end{align*}

For some random variable $x$, let $\E[x]$ denote its expected value.

\section{Derivative}
\subsection{Vector Gradient}
\begin{equation}
\nabla_{\bmx} \bmy = [\frac{\partial \bmy}{\partial x_1}, \ldots, \frac{\partial \bmy}{\partial x_N}]
\label{eq:Derivative-random_vector}
\end{equation}

\section{Determinant Operator}
\subsection{Random Properties}
For scalar $c$ and $N\times N$ identity matrix $I$,
\[ \det(cI) = c^N. \]

\section{Trace Operator}
Defined for $N \times N$ square matrix $\bmA$ as
\begin{equation}
\Trace(\bmA) \Def \sum_{i}^{N} a_{ii}
\label{eq:Definition-Trace}
\end{equation}

\subsection{Properties}

\subsubsection{$\Trace(c\bmA + d\bmB) = c\Trace(\bmA) + d\Trace(\bmB)$}
For scalars $c$ and $d$, square matrices $\bmA$ and $\bmB$.

\subsubsection{$\Trace(\bmA\bmB) = \Trace(\bmB\bmA) = \Trace(\bmA^\top\bmB) = \Trace(\bmA\bmB^\top) = \sum_{i,j} a_{ij}b_{ij}$}
\label{sec:trace-communitivity}
And clearly, also $\Trace(\bmB^\top\bmA) = \Trace(\bmB\bmA^\top) = \sum_{i,j} a_{ij}b_{ij} = \Trace(\bmA\bmB)$.

\subsubsection{$\bmx^\top \bmA \bmx = \Trace(\bmA \bmx \bmx^\top)$}
This can be seen from
\[\bmx^\top \bmA \bmx = \left[\begin{array}{ccc}
\sum\limits_{i=1}^N x_i a_{i, 1} & \cdots & \sum\limits_{i=1}^N x_i a_{i, N}
\end{array}\right] \bmx = \sum_{j=1}^{N} x_j \sum_{i=1}^N x_i a_{i, j} = \sum_{j=1}^{N} \sum_{i=1}^N a_{i, j} (\bmx \bmx^\top)_{i,j} = \Trace(\bmA \bmx \bmx^\top). \]
The last equality follows from \cref{sec:trace-communitivity}. %, which also implies $\bmx^\top \bmA \bmx = \Trace(\bmx \bmx^\top \bmA)$.

\subsection{Derivatives}

\subsubsection{$\nabla_{\bmx}\Trace(\bmx \bmx^\top \bmA) = \bmx^\top (\bmA + \bmA^\top)$}
\label{sec:Trace-Derivative-tr(xx'A)}
For square matrix $\bmA$.
Note that $\bmx^\top (\bmA + \bmA^\top) = 2\bmx^\top \bmA$ for symmetric $\bmA$.

See \cref{proof:Trace-Derivative-tr(xx'A)} for proof.

\subsection{Relation to Determinant}


\section{Expected Values}
For $\bmx \in \R^d$, with expected value $\E[\bmx] = \bmmu$ and covariance matrix $ \bmSigma = \E[(\bmx - \E[\bmx]) (\bmx - \E[\bmx])^\top]$,
\begin{align}
\E[x_i^2] &= \Sigma_{i,i} + \mu_i^2
\\
\E[\bmx^\top \bmA \bmx] &= \Trace(\bmA \bmSigma) + \mu^\top \bmA \mu
\\
\E_{x} \left[ (y - \bmx\top \bm{w})^2 \right]
&= (y - \bm{w}^\top \bmmu)^2 + \bm{w}^\top \bmSigma \bm{w}.
\end{align}

\appendix
\section{Proofs}
\subsection{Trace}
\subsubsection{\hyperref[sec:Trace-Derivative-tr(xx'A)]{$\nabla_{\bmx}\Trace(\bmx \bmx^\top \bmA) = \bmx^\top (\bmA + \bmA^\top)$}}
\label{proof:Trace-Derivative-tr(xx'A)}
This proof can likely be generalized to non-square matrixes (and possibly some communicativeness, given the flexibility afforded by the trace), but the restricted case is presented here.

For square $N\times N$ matrix $\bmA$,
\begin{align*}
\nabla_{\bmx}\Trace(\bmx \bmx^\top \bmA) =
\frac{d}{d \bmx}\Trace(\bmx \bmx^\top \bmA) &=
\frac{d}{d \bmx} \sum_{i}^{N} \sum_{k}^{N} x_i x_k a_{ik}.
\end{align*}
Recall \cref{eq:Derivative-random_vector}, and consider for any $j \in \{1,\ldots, N\}$:
\begin{align*}
\frac{\partial}{\partial x_j} \sum_{i}^{N} \sum_{k}^{N} x_i x_k a_{ik} &=
[x_1a_{1,j} + x_2a_{2,j} + \cdots + x_{j-1}a_{j-1,j} + x_{j+1}a_{j+1,j} + \cdots x_N a_{N,j}] \\ &\qquad + \frac{\partial}{\partial x_j} \sum_{k}^{N} x_j x_k a_{jk} \\
&= \left[\sum_i^N x_ia_{ij} - x_ja_{jj}\right] + \sum_{k}^{N} x_k a_{jk} - x_ja_{jj} + \frac{\partial}{\partial x_j} x_j x_j a_{jj} \\
&= \sum_i^N x_ia_{ij} + \sum_{k}^{N} x_k a_{jk} - 2 x_ja_{jj} + 2 x_ja_{jj} \\
&= \bmx^\top \bma_j + \bmx^\top [\bma^\top]_j,
\end{align*}
where $[\bma^\top]_j$ is the $j$th column of $A^\top$.

This equally applies for any $j$ in $1\ldots N$, and so for the full gradient:
\begin{align*}
\nabla_{\bmx}\Trace(\bmx \bmx^\top \bmA) =
\frac{d}{d \bmx} \sum_{i}^{N} \sum_{k}^{N} x_i x_k a_{ik} &=
[\bmx^\top \bma_1 \cdots \bmx^\top \bma_N] + [\bmx^\top [\bma^\top]_1 \cdots \bmx^\top [\bma^\top]_N] \\
= \bmx^\top (\bmA + \bmA^\top).
\end{align*}

\end{document}
